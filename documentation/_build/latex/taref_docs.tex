% Generated by Sphinx.
\def\sphinxdocclass{report}
\documentclass[letterpaper,10pt,english]{sphinxmanual}
\usepackage[utf8]{inputenc}
\DeclareUnicodeCharacter{00A0}{\nobreakspace}
\usepackage{cmap}
\usepackage[T1]{fontenc}
\usepackage{babel}
\usepackage{times}
\usepackage[Bjarne]{fncychap}
\usepackage{longtable}
\usepackage{sphinx}
\usepackage{multirow}

\addto\captionsenglish{\renewcommand{\figurename}{Fig. }}
\addto\captionsenglish{\renewcommand{\tablename}{Table }}
\floatname{literal-block}{Listing }



\title{taref\_docs Documentation}
\date{February 11, 2016}
\release{1.0}
\author{Thomas Aref}
\newcommand{\sphinxlogo}{}
\renewcommand{\releasename}{Release}
\makeindex

\makeatletter
\def\PYG@reset{\let\PYG@it=\relax \let\PYG@bf=\relax%
    \let\PYG@ul=\relax \let\PYG@tc=\relax%
    \let\PYG@bc=\relax \let\PYG@ff=\relax}
\def\PYG@tok#1{\csname PYG@tok@#1\endcsname}
\def\PYG@toks#1+{\ifx\relax#1\empty\else%
    \PYG@tok{#1}\expandafter\PYG@toks\fi}
\def\PYG@do#1{\PYG@bc{\PYG@tc{\PYG@ul{%
    \PYG@it{\PYG@bf{\PYG@ff{#1}}}}}}}
\def\PYG#1#2{\PYG@reset\PYG@toks#1+\relax+\PYG@do{#2}}

\expandafter\def\csname PYG@tok@gd\endcsname{\def\PYG@tc##1{\textcolor[rgb]{0.63,0.00,0.00}{##1}}}
\expandafter\def\csname PYG@tok@gu\endcsname{\let\PYG@bf=\textbf\def\PYG@tc##1{\textcolor[rgb]{0.50,0.00,0.50}{##1}}}
\expandafter\def\csname PYG@tok@gt\endcsname{\def\PYG@tc##1{\textcolor[rgb]{0.00,0.27,0.87}{##1}}}
\expandafter\def\csname PYG@tok@gs\endcsname{\let\PYG@bf=\textbf}
\expandafter\def\csname PYG@tok@gr\endcsname{\def\PYG@tc##1{\textcolor[rgb]{1.00,0.00,0.00}{##1}}}
\expandafter\def\csname PYG@tok@cm\endcsname{\let\PYG@it=\textit\def\PYG@tc##1{\textcolor[rgb]{0.25,0.50,0.56}{##1}}}
\expandafter\def\csname PYG@tok@vg\endcsname{\def\PYG@tc##1{\textcolor[rgb]{0.73,0.38,0.84}{##1}}}
\expandafter\def\csname PYG@tok@m\endcsname{\def\PYG@tc##1{\textcolor[rgb]{0.13,0.50,0.31}{##1}}}
\expandafter\def\csname PYG@tok@mh\endcsname{\def\PYG@tc##1{\textcolor[rgb]{0.13,0.50,0.31}{##1}}}
\expandafter\def\csname PYG@tok@cs\endcsname{\def\PYG@tc##1{\textcolor[rgb]{0.25,0.50,0.56}{##1}}\def\PYG@bc##1{\setlength{\fboxsep}{0pt}\colorbox[rgb]{1.00,0.94,0.94}{\strut ##1}}}
\expandafter\def\csname PYG@tok@ge\endcsname{\let\PYG@it=\textit}
\expandafter\def\csname PYG@tok@vc\endcsname{\def\PYG@tc##1{\textcolor[rgb]{0.73,0.38,0.84}{##1}}}
\expandafter\def\csname PYG@tok@il\endcsname{\def\PYG@tc##1{\textcolor[rgb]{0.13,0.50,0.31}{##1}}}
\expandafter\def\csname PYG@tok@go\endcsname{\def\PYG@tc##1{\textcolor[rgb]{0.20,0.20,0.20}{##1}}}
\expandafter\def\csname PYG@tok@cp\endcsname{\def\PYG@tc##1{\textcolor[rgb]{0.00,0.44,0.13}{##1}}}
\expandafter\def\csname PYG@tok@gi\endcsname{\def\PYG@tc##1{\textcolor[rgb]{0.00,0.63,0.00}{##1}}}
\expandafter\def\csname PYG@tok@gh\endcsname{\let\PYG@bf=\textbf\def\PYG@tc##1{\textcolor[rgb]{0.00,0.00,0.50}{##1}}}
\expandafter\def\csname PYG@tok@ni\endcsname{\let\PYG@bf=\textbf\def\PYG@tc##1{\textcolor[rgb]{0.84,0.33,0.22}{##1}}}
\expandafter\def\csname PYG@tok@nl\endcsname{\let\PYG@bf=\textbf\def\PYG@tc##1{\textcolor[rgb]{0.00,0.13,0.44}{##1}}}
\expandafter\def\csname PYG@tok@nn\endcsname{\let\PYG@bf=\textbf\def\PYG@tc##1{\textcolor[rgb]{0.05,0.52,0.71}{##1}}}
\expandafter\def\csname PYG@tok@no\endcsname{\def\PYG@tc##1{\textcolor[rgb]{0.38,0.68,0.84}{##1}}}
\expandafter\def\csname PYG@tok@na\endcsname{\def\PYG@tc##1{\textcolor[rgb]{0.25,0.44,0.63}{##1}}}
\expandafter\def\csname PYG@tok@nb\endcsname{\def\PYG@tc##1{\textcolor[rgb]{0.00,0.44,0.13}{##1}}}
\expandafter\def\csname PYG@tok@nc\endcsname{\let\PYG@bf=\textbf\def\PYG@tc##1{\textcolor[rgb]{0.05,0.52,0.71}{##1}}}
\expandafter\def\csname PYG@tok@nd\endcsname{\let\PYG@bf=\textbf\def\PYG@tc##1{\textcolor[rgb]{0.33,0.33,0.33}{##1}}}
\expandafter\def\csname PYG@tok@ne\endcsname{\def\PYG@tc##1{\textcolor[rgb]{0.00,0.44,0.13}{##1}}}
\expandafter\def\csname PYG@tok@nf\endcsname{\def\PYG@tc##1{\textcolor[rgb]{0.02,0.16,0.49}{##1}}}
\expandafter\def\csname PYG@tok@si\endcsname{\let\PYG@it=\textit\def\PYG@tc##1{\textcolor[rgb]{0.44,0.63,0.82}{##1}}}
\expandafter\def\csname PYG@tok@s2\endcsname{\def\PYG@tc##1{\textcolor[rgb]{0.25,0.44,0.63}{##1}}}
\expandafter\def\csname PYG@tok@vi\endcsname{\def\PYG@tc##1{\textcolor[rgb]{0.73,0.38,0.84}{##1}}}
\expandafter\def\csname PYG@tok@nt\endcsname{\let\PYG@bf=\textbf\def\PYG@tc##1{\textcolor[rgb]{0.02,0.16,0.45}{##1}}}
\expandafter\def\csname PYG@tok@nv\endcsname{\def\PYG@tc##1{\textcolor[rgb]{0.73,0.38,0.84}{##1}}}
\expandafter\def\csname PYG@tok@s1\endcsname{\def\PYG@tc##1{\textcolor[rgb]{0.25,0.44,0.63}{##1}}}
\expandafter\def\csname PYG@tok@gp\endcsname{\let\PYG@bf=\textbf\def\PYG@tc##1{\textcolor[rgb]{0.78,0.36,0.04}{##1}}}
\expandafter\def\csname PYG@tok@sh\endcsname{\def\PYG@tc##1{\textcolor[rgb]{0.25,0.44,0.63}{##1}}}
\expandafter\def\csname PYG@tok@ow\endcsname{\let\PYG@bf=\textbf\def\PYG@tc##1{\textcolor[rgb]{0.00,0.44,0.13}{##1}}}
\expandafter\def\csname PYG@tok@sx\endcsname{\def\PYG@tc##1{\textcolor[rgb]{0.78,0.36,0.04}{##1}}}
\expandafter\def\csname PYG@tok@bp\endcsname{\def\PYG@tc##1{\textcolor[rgb]{0.00,0.44,0.13}{##1}}}
\expandafter\def\csname PYG@tok@c1\endcsname{\let\PYG@it=\textit\def\PYG@tc##1{\textcolor[rgb]{0.25,0.50,0.56}{##1}}}
\expandafter\def\csname PYG@tok@kc\endcsname{\let\PYG@bf=\textbf\def\PYG@tc##1{\textcolor[rgb]{0.00,0.44,0.13}{##1}}}
\expandafter\def\csname PYG@tok@c\endcsname{\let\PYG@it=\textit\def\PYG@tc##1{\textcolor[rgb]{0.25,0.50,0.56}{##1}}}
\expandafter\def\csname PYG@tok@mf\endcsname{\def\PYG@tc##1{\textcolor[rgb]{0.13,0.50,0.31}{##1}}}
\expandafter\def\csname PYG@tok@err\endcsname{\def\PYG@bc##1{\setlength{\fboxsep}{0pt}\fcolorbox[rgb]{1.00,0.00,0.00}{1,1,1}{\strut ##1}}}
\expandafter\def\csname PYG@tok@mb\endcsname{\def\PYG@tc##1{\textcolor[rgb]{0.13,0.50,0.31}{##1}}}
\expandafter\def\csname PYG@tok@ss\endcsname{\def\PYG@tc##1{\textcolor[rgb]{0.32,0.47,0.09}{##1}}}
\expandafter\def\csname PYG@tok@sr\endcsname{\def\PYG@tc##1{\textcolor[rgb]{0.14,0.33,0.53}{##1}}}
\expandafter\def\csname PYG@tok@mo\endcsname{\def\PYG@tc##1{\textcolor[rgb]{0.13,0.50,0.31}{##1}}}
\expandafter\def\csname PYG@tok@kd\endcsname{\let\PYG@bf=\textbf\def\PYG@tc##1{\textcolor[rgb]{0.00,0.44,0.13}{##1}}}
\expandafter\def\csname PYG@tok@mi\endcsname{\def\PYG@tc##1{\textcolor[rgb]{0.13,0.50,0.31}{##1}}}
\expandafter\def\csname PYG@tok@kn\endcsname{\let\PYG@bf=\textbf\def\PYG@tc##1{\textcolor[rgb]{0.00,0.44,0.13}{##1}}}
\expandafter\def\csname PYG@tok@o\endcsname{\def\PYG@tc##1{\textcolor[rgb]{0.40,0.40,0.40}{##1}}}
\expandafter\def\csname PYG@tok@kr\endcsname{\let\PYG@bf=\textbf\def\PYG@tc##1{\textcolor[rgb]{0.00,0.44,0.13}{##1}}}
\expandafter\def\csname PYG@tok@s\endcsname{\def\PYG@tc##1{\textcolor[rgb]{0.25,0.44,0.63}{##1}}}
\expandafter\def\csname PYG@tok@kp\endcsname{\def\PYG@tc##1{\textcolor[rgb]{0.00,0.44,0.13}{##1}}}
\expandafter\def\csname PYG@tok@w\endcsname{\def\PYG@tc##1{\textcolor[rgb]{0.73,0.73,0.73}{##1}}}
\expandafter\def\csname PYG@tok@kt\endcsname{\def\PYG@tc##1{\textcolor[rgb]{0.56,0.13,0.00}{##1}}}
\expandafter\def\csname PYG@tok@sc\endcsname{\def\PYG@tc##1{\textcolor[rgb]{0.25,0.44,0.63}{##1}}}
\expandafter\def\csname PYG@tok@sb\endcsname{\def\PYG@tc##1{\textcolor[rgb]{0.25,0.44,0.63}{##1}}}
\expandafter\def\csname PYG@tok@k\endcsname{\let\PYG@bf=\textbf\def\PYG@tc##1{\textcolor[rgb]{0.00,0.44,0.13}{##1}}}
\expandafter\def\csname PYG@tok@se\endcsname{\let\PYG@bf=\textbf\def\PYG@tc##1{\textcolor[rgb]{0.25,0.44,0.63}{##1}}}
\expandafter\def\csname PYG@tok@sd\endcsname{\let\PYG@it=\textit\def\PYG@tc##1{\textcolor[rgb]{0.25,0.44,0.63}{##1}}}

\def\PYGZbs{\char`\\}
\def\PYGZus{\char`\_}
\def\PYGZob{\char`\{}
\def\PYGZcb{\char`\}}
\def\PYGZca{\char`\^}
\def\PYGZam{\char`\&}
\def\PYGZlt{\char`\<}
\def\PYGZgt{\char`\>}
\def\PYGZsh{\char`\#}
\def\PYGZpc{\char`\%}
\def\PYGZdl{\char`\$}
\def\PYGZhy{\char`\-}
\def\PYGZsq{\char`\'}
\def\PYGZdq{\char`\"}
\def\PYGZti{\char`\~}
% for compatibility with earlier versions
\def\PYGZat{@}
\def\PYGZlb{[}
\def\PYGZrb{]}
\makeatother

\renewcommand\PYGZsq{\textquotesingle}

\begin{document}

\maketitle
\tableofcontents
\phantomsection\label{index::doc}


Contents:


\chapter{Welcome to taref's documentation!}
\label{get_started/index::doc}\label{get_started/index:welcome-to-taref-s-documentation}
Contents:


\section{Getting Started}
\label{get_started/getting_started:getting-started}\label{get_started/getting_started::doc}
The taref package strives to make quick, easy-to-use, auto-display GUIs
with the option to extend them to custom GUIs at a later date. To do this,
taref is built on enaml, a programming language extension to python and framework
for creating professional user interfaces, and Atom, a framework for
creating memory efficient Python objects with enhanced features
such as dynamic initialization, validation, and
change notification for object attributes (similar in behavior to Enthought's Traits)

Basically,
taref generates dynamic enaml templates from a minimal information Atom
class with the option to later substitute these dynamic enaml templates
\begin{quote}

with enaml written specifically for the class.
\end{quote}

For example, some code that makes use of taref's shower function might look like this:

\begin{Verbatim}[commandchars=\\\{\}]
\PYG{k+kn}{from} \PYG{n+nn}{atom.api} \PYG{k+kn}{import} \PYG{n}{Atom}\PYG{p}{,} \PYG{n}{Float}\PYG{p}{,} \PYG{n}{Unicode}
\PYG{k+kn}{from} \PYG{n+nn}{taref.core.shower} \PYG{k+kn}{import} \PYG{n}{shower}

\PYG{k}{class} \PYG{n+nc}{Test}\PYG{p}{(}\PYG{n}{Atom}\PYG{p}{)}\PYG{p}{:}
    \PYG{n}{a}\PYG{o}{=}\PYG{n}{Float}\PYG{p}{(}\PYG{p}{)}
    \PYG{n}{b}\PYG{o}{=}\PYG{n}{Unicode}\PYG{p}{(}\PYG{p}{)}

\PYG{n}{t}\PYG{o}{=}\PYG{n}{Test}\PYG{p}{(}\PYG{p}{)}
\PYG{n}{shower}\PYG{p}{(}\PYG{n}{t}\PYG{p}{)}
\end{Verbatim}

and these few lines of code are all that is needed to produce a simple GUI that shows a and b in our
Test object t!

So what is happening?
First, we are an Atom class. Atom class are very similar to python's
regular classes. Something equivalent to our Test class above would be:

\begin{Verbatim}[commandchars=\\\{\}]
\PYG{k}{class} \PYG{n+nc}{Test}\PYG{p}{(}\PYG{n+nb}{object}\PYG{p}{)}\PYG{p}{:}
    \PYG{k}{def} \PYG{n+nf}{\PYGZus{}\PYGZus{}init\PYGZus{}\PYGZus{}}\PYG{p}{(}\PYG{n+nb+bp}{self}\PYG{p}{,} \PYG{n}{a}\PYG{o}{=}\PYG{l+m+mf}{0.0}\PYG{p}{,} \PYG{n}{b}\PYG{o}{=}\PYG{l+s}{\PYGZdq{}}\PYG{l+s}{\PYGZdq{}}\PYG{p}{)}\PYG{p}{:}
        \PYG{n+nb+bp}{self}\PYG{o}{.}\PYG{n}{a}\PYG{o}{=}\PYG{n}{a}
        \PYG{n+nb+bp}{self}\PYG{o}{.}\PYG{n}{b}\PYG{o}{=}\PYG{n}{b}
\end{Verbatim}

However, Atom provides some key advantages to using the above class for GUI making
First, in the Atom class, the type of a is fixed to being a float so
the GUI always knows how to display it. The members of Test are likewise fixed
so that none are added dynamically later.
Secondly, Atom can detect changes changes to it's members. To see this, we
look at the following code:

\begin{Verbatim}[commandchars=\\\{\}]
\PYG{k}{class} \PYG{n+nc}{Test}\PYG{p}{(}\PYG{n}{Atom}\PYG{p}{)}\PYG{p}{:}
    \PYG{n}{a}\PYG{o}{=}\PYG{n}{Float}\PYG{p}{(}\PYG{p}{)}
    \PYG{n}{b}\PYG{o}{=}\PYG{n}{Unicode}\PYG{p}{(}\PYG{p}{)}

    \PYG{k}{def} \PYG{n+nf}{\PYGZus{}observe\PYGZus{}a}\PYG{p}{(}\PYG{n+nb+bp}{self}\PYG{p}{,} \PYG{n}{change}\PYG{p}{)}\PYG{p}{:}
        \PYG{k}{print} \PYG{n}{change}
\end{Verbatim}

Now every time variable a is changed, in the GUI or in code,
it will print out that change.
The final advantage of Atom is that metadata can be added to the variable.
For example,

\begin{Verbatim}[commandchars=\\\{\}]
\PYG{n}{t}\PYG{o}{.}\PYG{n}{a}\PYG{o}{=}\PYG{l+m+mf}{4.0}
\PYG{n}{t}\PYG{o}{.}\PYG{n}{get\PYGZus{}member}\PYG{p}{(}\PYG{l+s}{\PYGZdq{}}\PYG{l+s}{a}\PYG{l+s}{\PYGZdq{}}\PYG{p}{)}\PYG{o}{.}\PYG{n}{tag}\PYG{p}{(}\PYG{n}{label}\PYG{o}{=}\PYG{l+s}{\PYGZdq{}}\PYG{l+s}{My Float}\PYG{l+s}{\PYGZdq{}}\PYG{p}{)}
\PYG{k}{print} \PYG{n}{t}\PYG{o}{.}\PYG{n}{a}
\PYG{k}{print} \PYG{n}{t}\PYG{o}{.}\PYG{n}{get\PYGZus{}member}\PYG{p}{(}\PYG{l+s}{\PYGZdq{}}\PYG{l+s}{a}\PYG{l+s}{\PYGZdq{}}\PYG{p}{)}\PYG{o}{.}\PYG{n}{metadata}
\end{Verbatim}

Combining this with the functionality with the shower function:

\begin{Verbatim}[commandchars=\\\{\}]
\PYG{k+kn}{from} \PYG{n+nn}{atom.api} \PYG{k+kn}{import} \PYG{n}{Atom}\PYG{p}{,} \PYG{n}{Float}\PYG{p}{,} \PYG{n}{Unicode}
\PYG{k+kn}{from} \PYG{n+nn}{taref.core.shower} \PYG{k+kn}{import} \PYG{n}{shower}

\PYG{k}{class} \PYG{n+nc}{Test}\PYG{p}{(}\PYG{n}{Atom}\PYG{p}{)}\PYG{p}{:}
    \PYG{n}{a}\PYG{o}{=}\PYG{n}{Float}\PYG{p}{(}\PYG{p}{)}\PYG{o}{.}\PYG{n}{tag}\PYG{p}{(}\PYG{n}{label}\PYG{o}{=}\PYG{l+s}{\PYGZdq{}}\PYG{l+s}{My Float}\PYG{l+s}{\PYGZdq{}}\PYG{p}{)}
    \PYG{n}{b}\PYG{o}{=}\PYG{n}{Unicode}\PYG{p}{(}\PYG{p}{)}

    \PYG{k}{def} \PYG{n+nf}{\PYGZus{}observe\PYGZus{}a}\PYG{p}{(}\PYG{n+nb+bp}{self}\PYG{p}{,} \PYG{n}{change}\PYG{p}{)}\PYG{p}{:}
        \PYG{k}{print} \PYG{n}{change}

\PYG{n}{t}\PYG{o}{=}\PYG{n}{Test}\PYG{p}{(}\PYG{p}{)}
\PYG{n}{shower}\PYG{p}{(}\PYG{n}{t}\PYG{p}{)}
\end{Verbatim}

auto creates a GUI where a is now labelled ``My Float'' and every time a is changed
it is printed.

There are a number of custom tags defined in taref, such as ``label'', to give easy access to some
commonly used features. For example, suppose I wanted b to display as a multiline field rather than
a single line field:

\begin{Verbatim}[commandchars=\\\{\}]
\PYG{k}{class} \PYG{n+nc}{Test}\PYG{p}{(}\PYG{n}{Atom}\PYG{p}{)}\PYG{p}{:}
    \PYG{n}{a}\PYG{o}{=}\PYG{n}{Float}\PYG{p}{(}\PYG{p}{)}\PYG{o}{.}\PYG{n}{tag}\PYG{p}{(}\PYG{n}{label}\PYG{o}{=}\PYG{l+s}{\PYGZdq{}}\PYG{l+s}{My Float}\PYG{l+s}{\PYGZdq{}}\PYG{p}{)}
    \PYG{n}{b}\PYG{o}{=}\PYG{n}{Unicode}\PYG{p}{(}\PYG{p}{)}\PYG{o}{.}\PYG{n}{tag}\PYG{p}{(}\PYG{n}{spec}\PYG{o}{=}\PYG{l+s}{\PYGZdq{}}\PYG{l+s}{multiline}\PYG{l+s}{\PYGZdq{}}\PYG{p}{)}
\end{Verbatim}

In this case the spec tag allows quick access to a multiline field display.
Now suppose I want full control over the window that Test objects reside in
using the full power of enaml. I start an enaml file, ``test\_e.enaml'' that looks like this:

\begin{Verbatim}[commandchars=\\\{\}]
from enaml.widgets.api import MainWindow, Field, Label, HGroup

enamldef TestWindow(MainWindow):
    attr test
    HGroup:
        Label:
            text \PYGZlt{}\PYGZlt{} unicode(test.a)
        Field:
            text := b
\end{Verbatim}

In my python file, ``test.py'', I add the necessary pieces:

\begin{Verbatim}[commandchars=\\\{\}]
\PYG{k+kn}{from} \PYG{n+nn}{atom.api} \PYG{k+kn}{import} \PYG{n}{Atom}\PYG{p}{,} \PYG{n}{Float}\PYG{p}{,} \PYG{n}{Unicode}\PYG{p}{,} \PYG{n}{cached\PYGZus{}property}
\PYG{k+kn}{from} \PYG{n+nn}{taref.core.shower} \PYG{k+kn}{import} \PYG{n}{shower}
\PYG{k+kn}{from} \PYG{n+nn}{enaml} \PYG{k+kn}{import} \PYG{n}{imports}
\PYG{k}{with} \PYG{n}{imports}\PYG{p}{(}\PYG{p}{)}\PYG{p}{:}
    \PYG{k+kn}{from} \PYG{n+nn}{test\PYGZus{}e} \PYG{k+kn}{import} \PYG{n}{TestWindow}

\PYG{k}{class} \PYG{n+nc}{Test}\PYG{p}{(}\PYG{n}{Atom}\PYG{p}{)}\PYG{p}{:}
    \PYG{n}{a}\PYG{o}{=}\PYG{n}{Float}\PYG{p}{(}\PYG{p}{)}\PYG{o}{.}\PYG{n}{tag}\PYG{p}{(}\PYG{n}{label}\PYG{o}{=}\PYG{l+s}{\PYGZdq{}}\PYG{l+s}{My Float}\PYG{l+s}{\PYGZdq{}}\PYG{p}{)}
    \PYG{n}{b}\PYG{o}{=}\PYG{n}{Unicode}\PYG{p}{(}\PYG{p}{)}

    \PYG{k}{def} \PYG{n+nf}{\PYGZus{}observe\PYGZus{}a}\PYG{p}{(}\PYG{n+nb+bp}{self}\PYG{p}{,} \PYG{n}{change}\PYG{p}{)}\PYG{p}{:}
        \PYG{k}{print} \PYG{n}{change}

    \PYG{n+nd}{@cached\PYGZus{}property}
    \PYG{k}{def} \PYG{n+nf}{view\PYGZus{}window}\PYG{p}{(}\PYG{n+nb+bp}{self}\PYG{p}{)}\PYG{p}{:}
        \PYG{k}{return} \PYG{n}{TestWindow}\PYG{p}{(}\PYG{n}{test}\PYG{o}{=}\PYG{n+nb+bp}{self}\PYG{p}{)}

\PYG{n}{t}\PYG{o}{=}\PYG{n}{Test}\PYG{p}{(}\PYG{p}{)}
\PYG{n}{shower}\PYG{p}{(}\PYG{n}{t}\PYG{p}{)}
\end{Verbatim}

and now I have replaced the default dynamic view of Test with a custom one,
while still keeping it compatible with the rest of taref's framework!


\chapter{Introduction:}
\label{introduction:introduction}\label{introduction::doc}

\section{Other header}
\label{introduction:other-header}
How does this work:

\begin{Verbatim}[commandchars=\\\{\}]
recognize code
print \PYGZdq{}yo\PYGZdq{}
\end{Verbatim}

Maybe

can I see this


\chapter{Let's try lists:}
\label{introduction:reftest}\label{introduction:let-s-try-lists}\begin{itemize}
\item {} 
item 1

\item {} 
item 2

\item {} 
item 3

\end{itemize}
\begin{enumerate}
\item {} 
item 1

\item {} 
item 2

\item {} 
item 3

\end{enumerate}


\chapter{Let's try tables:}
\label{introduction:let-s-try-tables}
\emph{italics}, \textbf{bold}, or \code{monotype}.

\begin{tabulary}{\linewidth}{|L|L|}
\hline
\textsf{\relax 
Name
} & \textsf{\relax 
age
}\\
\hline
item1
 & 
12
\\
\hline
item2
 & 
42
\\
\hline
item3
 & 
3
\\
\hline\end{tabulary}


\begin{Verbatim}[commandchars=\\\{\}]
Introduction:
==============

*****************
Other header
*****************

How does this work::

    recognize code
    print \PYGZdq{}yo\PYGZdq{}

Maybe


can I see this

.. \PYGZus{}reftest:

Let\PYGZsq{}s try lists:
================

*  item 1
* item 2
* item 3

\PYGZsh{}. item 1
\PYGZsh{}. item 2
\PYGZsh{}. item 3

Let\PYGZsq{}s try tables:
=================
*italics*, **bold**, or {}`{}`monotype{}`{}`.

========  ====
Name      age
========  ====
item1     12
item2     42
item3     3
========  ====

.. literalinclude:: introduction.rst

does a reference work? :ref:{}`reftest{}`

.. function:: enumerate(sequence[, start=0])

   Return an iterator that yields tuples of an index and an item of the
   *sequence*. (And so on.)

.. autofunction::taref.core.atom\PYGZus{}extension.get\PYGZus{}tag
\end{Verbatim}

does a reference work? {\hyperref[introduction:reftest]{\emph{\DUspan{}{Let's try lists:}}}}
\index{enumerate() (built-in function)}

\begin{fulllineitems}
\phantomsection\label{introduction:enumerate}\pysiglinewithargsret{\bfcode{enumerate}}{\emph{sequence}\optional{, \emph{start=0}}}{}
Return an iterator that yields tuples of an index and an item of the
\emph{sequence}. (And so on.)

\end{fulllineitems}



\chapter{Getting Started}
\label{getting_started:getting-started}\label{getting_started::doc}
The taref package strives to make quick, easy-to-use, auto-display GUIs
with the option to extend them to custom GUIs at a later date. To do this,
taref is built on enaml, a programming language extension to python and framework
for creating professional user interfaces, and Atom, a framework for
creating memory efficient Python objects with enhanced features
such as dynamic initialization, validation, and
change notification for object attributes (similar in behavior to Enthought's Traits)

Basically,
taref generates dynamic enaml templates from a minimal information Atom
class with the option to later substitute these dynamic enaml templates
\begin{quote}

with enaml written specifically for the class.
\end{quote}

For example, some code that makes use of taref's shower function might look like this:

\begin{Verbatim}[commandchars=\\\{\}]
\PYG{k+kn}{from} \PYG{n+nn}{atom.api} \PYG{k+kn}{import} \PYG{n}{Atom}\PYG{p}{,} \PYG{n}{Float}\PYG{p}{,} \PYG{n}{Unicode}
\PYG{k+kn}{from} \PYG{n+nn}{taref.core.shower} \PYG{k+kn}{import} \PYG{n}{shower}

\PYG{k}{class} \PYG{n+nc}{Test}\PYG{p}{(}\PYG{n}{Atom}\PYG{p}{)}\PYG{p}{:}
    \PYG{n}{a}\PYG{o}{=}\PYG{n}{Float}\PYG{p}{(}\PYG{p}{)}
    \PYG{n}{b}\PYG{o}{=}\PYG{n}{Unicode}\PYG{p}{(}\PYG{p}{)}

\PYG{n}{t}\PYG{o}{=}\PYG{n}{Test}\PYG{p}{(}\PYG{p}{)}
\PYG{n}{shower}\PYG{p}{(}\PYG{n}{t}\PYG{p}{)}
\end{Verbatim}

and these few lines of code are all that is needed to produce a simple GUI that shows a and b in our
Test object t!

So what is happening?
First, we are an Atom class. Atom class are very similar to python's
regular classes. Something equivalent to our Test class above would be:

\begin{Verbatim}[commandchars=\\\{\}]
\PYG{k}{class} \PYG{n+nc}{Test}\PYG{p}{(}\PYG{n+nb}{object}\PYG{p}{)}\PYG{p}{:}
    \PYG{k}{def} \PYG{n+nf}{\PYGZus{}\PYGZus{}init\PYGZus{}\PYGZus{}}\PYG{p}{(}\PYG{n+nb+bp}{self}\PYG{p}{,} \PYG{n}{a}\PYG{o}{=}\PYG{l+m+mf}{0.0}\PYG{p}{,} \PYG{n}{b}\PYG{o}{=}\PYG{l+s}{\PYGZdq{}}\PYG{l+s}{\PYGZdq{}}\PYG{p}{)}\PYG{p}{:}
        \PYG{n+nb+bp}{self}\PYG{o}{.}\PYG{n}{a}\PYG{o}{=}\PYG{n}{a}
        \PYG{n+nb+bp}{self}\PYG{o}{.}\PYG{n}{b}\PYG{o}{=}\PYG{n}{b}
\end{Verbatim}

However, Atom provides some key advantages to using the above class for GUI making
First, in the Atom class, the type of a is fixed to being a float so
the GUI always knows how to display it. The members of Test are likewise fixed
so that none are added dynamically later.
Secondly, Atom can detect changes changes to it's members. To see this, we
look at the following code:

\begin{Verbatim}[commandchars=\\\{\}]
\PYG{k}{class} \PYG{n+nc}{Test}\PYG{p}{(}\PYG{n}{Atom}\PYG{p}{)}\PYG{p}{:}
    \PYG{n}{a}\PYG{o}{=}\PYG{n}{Float}\PYG{p}{(}\PYG{p}{)}
    \PYG{n}{b}\PYG{o}{=}\PYG{n}{Unicode}\PYG{p}{(}\PYG{p}{)}

    \PYG{k}{def} \PYG{n+nf}{\PYGZus{}observe\PYGZus{}a}\PYG{p}{(}\PYG{n+nb+bp}{self}\PYG{p}{,} \PYG{n}{change}\PYG{p}{)}\PYG{p}{:}
        \PYG{k}{print} \PYG{n}{change}
\end{Verbatim}

Now every time variable a is changed, in the GUI or in code,
it will print out that change.
The final advantage of Atom is that metadata can be added to the variable.
For example,

\begin{Verbatim}[commandchars=\\\{\}]
\PYG{n}{t}\PYG{o}{.}\PYG{n}{a}\PYG{o}{=}\PYG{l+m+mf}{4.0}
\PYG{n}{t}\PYG{o}{.}\PYG{n}{get\PYGZus{}member}\PYG{p}{(}\PYG{l+s}{\PYGZdq{}}\PYG{l+s}{a}\PYG{l+s}{\PYGZdq{}}\PYG{p}{)}\PYG{o}{.}\PYG{n}{tag}\PYG{p}{(}\PYG{n}{label}\PYG{o}{=}\PYG{l+s}{\PYGZdq{}}\PYG{l+s}{My Float}\PYG{l+s}{\PYGZdq{}}\PYG{p}{)}
\PYG{k}{print} \PYG{n}{t}\PYG{o}{.}\PYG{n}{a}
\PYG{k}{print} \PYG{n}{t}\PYG{o}{.}\PYG{n}{get\PYGZus{}member}\PYG{p}{(}\PYG{l+s}{\PYGZdq{}}\PYG{l+s}{a}\PYG{l+s}{\PYGZdq{}}\PYG{p}{)}\PYG{o}{.}\PYG{n}{metadata}
\end{Verbatim}

Combining this with the functionality with the shower function:

\begin{Verbatim}[commandchars=\\\{\}]
\PYG{k+kn}{from} \PYG{n+nn}{atom.api} \PYG{k+kn}{import} \PYG{n}{Atom}\PYG{p}{,} \PYG{n}{Float}\PYG{p}{,} \PYG{n}{Unicode}
\PYG{k+kn}{from} \PYG{n+nn}{taref.core.shower} \PYG{k+kn}{import} \PYG{n}{shower}

\PYG{k}{class} \PYG{n+nc}{Test}\PYG{p}{(}\PYG{n}{Atom}\PYG{p}{)}\PYG{p}{:}
    \PYG{n}{a}\PYG{o}{=}\PYG{n}{Float}\PYG{p}{(}\PYG{p}{)}\PYG{o}{.}\PYG{n}{tag}\PYG{p}{(}\PYG{n}{label}\PYG{o}{=}\PYG{l+s}{\PYGZdq{}}\PYG{l+s}{My Float}\PYG{l+s}{\PYGZdq{}}\PYG{p}{)}
    \PYG{n}{b}\PYG{o}{=}\PYG{n}{Unicode}\PYG{p}{(}\PYG{p}{)}

    \PYG{k}{def} \PYG{n+nf}{\PYGZus{}observe\PYGZus{}a}\PYG{p}{(}\PYG{n+nb+bp}{self}\PYG{p}{,} \PYG{n}{change}\PYG{p}{)}\PYG{p}{:}
        \PYG{k}{print} \PYG{n}{change}

\PYG{n}{t}\PYG{o}{=}\PYG{n}{Test}\PYG{p}{(}\PYG{p}{)}
\PYG{n}{shower}\PYG{p}{(}\PYG{n}{t}\PYG{p}{)}
\end{Verbatim}

auto creates a GUI where a is now labelled ``My Float'' and every time a is changed
it is printed.

There are a number of custom tags defined in taref, such as ``label'', to give easy access to some
commonly used features. For example, suppose I wanted b to display as a multiline field rather than
a single line field:

\begin{Verbatim}[commandchars=\\\{\}]
\PYG{k}{class} \PYG{n+nc}{Test}\PYG{p}{(}\PYG{n}{Atom}\PYG{p}{)}\PYG{p}{:}
    \PYG{n}{a}\PYG{o}{=}\PYG{n}{Float}\PYG{p}{(}\PYG{p}{)}\PYG{o}{.}\PYG{n}{tag}\PYG{p}{(}\PYG{n}{label}\PYG{o}{=}\PYG{l+s}{\PYGZdq{}}\PYG{l+s}{My Float}\PYG{l+s}{\PYGZdq{}}\PYG{p}{)}
    \PYG{n}{b}\PYG{o}{=}\PYG{n}{Unicode}\PYG{p}{(}\PYG{p}{)}\PYG{o}{.}\PYG{n}{tag}\PYG{p}{(}\PYG{n}{spec}\PYG{o}{=}\PYG{l+s}{\PYGZdq{}}\PYG{l+s}{multiline}\PYG{l+s}{\PYGZdq{}}\PYG{p}{)}
\end{Verbatim}

In this case the spec tag allows quick access to a multiline field display.
Now suppose I want full control over the window that Test objects reside in
using the full power of enaml. I start an enaml file, ``test\_e.enaml'' that looks like this:

\begin{Verbatim}[commandchars=\\\{\}]
from enaml.widgets.api import MainWindow, Field, Label, HGroup

enamldef TestWindow(MainWindow):
    attr test
    HGroup:
        Label:
            text \PYGZlt{}\PYGZlt{} unicode(test.a)
        Field:
            text := b
\end{Verbatim}

In my python file, ``test.py'', I add the necessary pieces:

\begin{Verbatim}[commandchars=\\\{\}]
\PYG{k+kn}{from} \PYG{n+nn}{atom.api} \PYG{k+kn}{import} \PYG{n}{Atom}\PYG{p}{,} \PYG{n}{Float}\PYG{p}{,} \PYG{n}{Unicode}\PYG{p}{,} \PYG{n}{cached\PYGZus{}property}
\PYG{k+kn}{from} \PYG{n+nn}{taref.core.shower} \PYG{k+kn}{import} \PYG{n}{shower}
\PYG{k+kn}{from} \PYG{n+nn}{enaml} \PYG{k+kn}{import} \PYG{n}{imports}
\PYG{k}{with} \PYG{n}{imports}\PYG{p}{(}\PYG{p}{)}\PYG{p}{:}
    \PYG{k+kn}{from} \PYG{n+nn}{test\PYGZus{}e} \PYG{k+kn}{import} \PYG{n}{TestWindow}

\PYG{k}{class} \PYG{n+nc}{Test}\PYG{p}{(}\PYG{n}{Atom}\PYG{p}{)}\PYG{p}{:}
    \PYG{n}{a}\PYG{o}{=}\PYG{n}{Float}\PYG{p}{(}\PYG{p}{)}\PYG{o}{.}\PYG{n}{tag}\PYG{p}{(}\PYG{n}{label}\PYG{o}{=}\PYG{l+s}{\PYGZdq{}}\PYG{l+s}{My Float}\PYG{l+s}{\PYGZdq{}}\PYG{p}{)}
    \PYG{n}{b}\PYG{o}{=}\PYG{n}{Unicode}\PYG{p}{(}\PYG{p}{)}

    \PYG{k}{def} \PYG{n+nf}{\PYGZus{}observe\PYGZus{}a}\PYG{p}{(}\PYG{n+nb+bp}{self}\PYG{p}{,} \PYG{n}{change}\PYG{p}{)}\PYG{p}{:}
        \PYG{k}{print} \PYG{n}{change}

    \PYG{n+nd}{@cached\PYGZus{}property}
    \PYG{k}{def} \PYG{n+nf}{view\PYGZus{}window}\PYG{p}{(}\PYG{n+nb+bp}{self}\PYG{p}{)}\PYG{p}{:}
        \PYG{k}{return} \PYG{n}{TestWindow}\PYG{p}{(}\PYG{n}{test}\PYG{o}{=}\PYG{n+nb+bp}{self}\PYG{p}{)}

\PYG{n}{t}\PYG{o}{=}\PYG{n}{Test}\PYG{p}{(}\PYG{p}{)}
\PYG{n}{shower}\PYG{p}{(}\PYG{n}{t}\PYG{p}{)}
\end{Verbatim}

and now I have replaced the default dynamic view of Test with a custom one,
while still keeping it compatible with the rest of taref's framework!


\chapter{taref}
\label{taref::doc}\label{taref:taref}
Contents:


\section{core}
\label{core_doc/core:core}\label{core_doc/core::doc}
Contents:


\subsection{atom\_extension}
\label{core_doc/atom_extension::doc}\label{core_doc/atom_extension:module-taref.core.atom_extension}\label{core_doc/atom_extension:atom-extension}\index{taref.core.atom\_extension (module)}
Created on Fri Jan 22 19:12:36 2016

@author: thomasaref

A collection of utility functions that extend Atom's functionality, used heavily by taref other modules.
To maintain compatibility with Atom classes, these are defined as standalone functions rather than extending the class.
Runtime also seem slightly better as standalone functions than as an extended class
\index{call\_func() (in module taref.core.atom\_extension)}

\begin{fulllineitems}
\phantomsection\label{core_doc/atom_extension:taref.core.atom_extension.call_func}\pysiglinewithargsret{\code{taref.core.atom\_extension.}\bfcode{call\_func}}{\emph{obj}, \emph{name}, \emph{**kwargs}}{}
calls a func using keyword assignments. If name corresponds to a Property, calls the get func.
otherwise, if name\_mangled func ``\_get\_''+name exists, calls that. Finally calls just the name if these are not the case

\end{fulllineitems}

\index{get\_all\_main\_params() (in module taref.core.atom\_extension)}

\begin{fulllineitems}
\phantomsection\label{core_doc/atom_extension:taref.core.atom_extension.get_all_main_params}\pysiglinewithargsret{\code{taref.core.atom\_extension.}\bfcode{get\_all\_main\_params}}{\emph{obj}}{}
all members in all\_params that are not tagged as sub.
Convenience function for more easily custom defining main\_params in child classes

\end{fulllineitems}

\index{get\_all\_params() (in module taref.core.atom\_extension)}

\begin{fulllineitems}
\phantomsection\label{core_doc/atom_extension:taref.core.atom_extension.get_all_params}\pysiglinewithargsret{\code{taref.core.atom\_extension.}\bfcode{get\_all\_params}}{\emph{obj}}{}
all members that are not tagged as private, i.e. not in reserved\_names and will behave as agents.
order of magnitude faster when combine with private\_property

\end{fulllineitems}

\index{get\_all\_tags() (in module taref.core.atom\_extension)}

\begin{fulllineitems}
\phantomsection\label{core_doc/atom_extension:taref.core.atom_extension.get_all_tags}\pysiglinewithargsret{\code{taref.core.atom\_extension.}\bfcode{get\_all\_tags}}{\emph{obj}, \emph{key}, \emph{key\_value=None}, \emph{none\_value=None}, \emph{search\_list=None}}{}
returns a list of names of parameters with a certain key\_value
Shortcut retrieve members with particular metadata. There are several variants based on inputs.
\begin{itemize}
\item {} 
With only obj and key specified, returns all member names who have that key

\item {} 
with key\_value specified, returns all member names that have that key set to key\_value

\item {} 
with key\_value and none\_value specified equal, returns all member names that have that key set to key\_value or do not have the tag

\item {} 
specifying search list limits the members searched

\item {} 
Finally, if key\_value is none, returns those members not matching none\_value

\end{itemize}

\end{fulllineitems}

\index{get\_inv() (in module taref.core.atom\_extension)}

\begin{fulllineitems}
\phantomsection\label{core_doc/atom_extension:taref.core.atom_extension.get_inv}\pysiglinewithargsret{\code{taref.core.atom\_extension.}\bfcode{get\_inv}}{\emph{obj}, \emph{name}, \emph{value}}{}
returns the inverse mapped value (meant for an Enum)

\end{fulllineitems}

\index{get\_main\_params() (in module taref.core.atom\_extension)}

\begin{fulllineitems}
\phantomsection\label{core_doc/atom_extension:taref.core.atom_extension.get_main_params}\pysiglinewithargsret{\code{taref.core.atom\_extension.}\bfcode{get\_main\_params}}{\emph{obj}}{}
returns main\_params if it exists and all possible main params if it does not

\end{fulllineitems}

\index{get\_map() (in module taref.core.atom\_extension)}

\begin{fulllineitems}
\phantomsection\label{core_doc/atom_extension:taref.core.atom_extension.get_map}\pysiglinewithargsret{\code{taref.core.atom\_extension.}\bfcode{get\_map}}{\emph{obj}, \emph{name}, \emph{value=None}, \emph{reset=False}}{}
gets the mapped value specified by the property mapping and returns the attribute value if it doesn't exist
gets the map of an Enum defined in the property name\_mapping or tag mapping.
value can be used to get the map for another value besides the Enum's current one.

\end{fulllineitems}

\index{get\_property\_names() (in module taref.core.atom\_extension)}

\begin{fulllineitems}
\phantomsection\label{core_doc/atom_extension:taref.core.atom_extension.get_property_names}\pysiglinewithargsret{\code{taref.core.atom\_extension.}\bfcode{get\_property\_names}}{\emph{obj}}{}
returns property names that are in all\_params

\end{fulllineitems}

\index{get\_property\_values() (in module taref.core.atom\_extension)}

\begin{fulllineitems}
\phantomsection\label{core_doc/atom_extension:taref.core.atom_extension.get_property_values}\pysiglinewithargsret{\code{taref.core.atom\_extension.}\bfcode{get\_property\_values}}{\emph{obj}}{}
returns property values that are in all\_params

\end{fulllineitems}

\index{get\_reserved\_names() (in module taref.core.atom\_extension)}

\begin{fulllineitems}
\phantomsection\label{core_doc/atom_extension:taref.core.atom_extension.get_reserved_names}\pysiglinewithargsret{\code{taref.core.atom\_extension.}\bfcode{get\_reserved\_names}}{\emph{obj}}{}
reserved names not to perform standard logging and display operations on,
i.e. members that are tagged as private and will behave as usual Atom members

\end{fulllineitems}

\index{get\_run\_params() (in module taref.core.atom\_extension)}

\begin{fulllineitems}
\phantomsection\label{core_doc/atom_extension:taref.core.atom_extension.get_run_params}\pysiglinewithargsret{\code{taref.core.atom\_extension.}\bfcode{get\_run\_params}}{\emph{f}, \emph{skip\_first=True}}{}
returns names of parameters a function will call, skips first parameter if skip\_first is True

\end{fulllineitems}

\index{get\_tag() (in module taref.core.atom\_extension)}

\begin{fulllineitems}
\phantomsection\label{core_doc/atom_extension:taref.core.atom_extension.get_tag}\pysiglinewithargsret{\code{taref.core.atom\_extension.}\bfcode{get\_tag}}{\emph{obj}, \emph{name}, \emph{key}, \emph{none\_value=None}}{}
Shortcut to retrieve metadata from an Atom member which also returns a none\_value if the metadata does not exist.
This is an easy way to get a tag on a particular member and provide a default if it isn't there.

\end{fulllineitems}

\index{get\_type() (in module taref.core.atom\_extension)}

\begin{fulllineitems}
\phantomsection\label{core_doc/atom_extension:taref.core.atom_extension.get_type}\pysiglinewithargsret{\code{taref.core.atom\_extension.}\bfcode{get\_type}}{\emph{obj}, \emph{name}}{}
returns type of member with given name, with possible override via tag typer

\end{fulllineitems}

\index{instancemethod (class in taref.core.atom\_extension)}

\begin{fulllineitems}
\phantomsection\label{core_doc/atom_extension:taref.core.atom_extension.instancemethod}\pysiglinewithargsret{\strong{class }\code{taref.core.atom\_extension.}\bfcode{instancemethod}}{\emph{obj}, \emph{name=None}}{}
disposable decorator object for instancemethods defined outside of Atom class

\end{fulllineitems}

\index{log\_func() (in module taref.core.atom\_extension)}

\begin{fulllineitems}
\phantomsection\label{core_doc/atom_extension:taref.core.atom_extension.log_func}\pysiglinewithargsret{\code{taref.core.atom\_extension.}\bfcode{log\_func}}{\emph{func}, \emph{pname=None}}{}
logging decorator for Callables that logs call if tag log!=False

\end{fulllineitems}

\index{lowhigh\_check() (in module taref.core.atom\_extension)}

\begin{fulllineitems}
\phantomsection\label{core_doc/atom_extension:taref.core.atom_extension.lowhigh_check}\pysiglinewithargsret{\code{taref.core.atom\_extension.}\bfcode{lowhigh\_check}}{\emph{obj}, \emph{name}, \emph{value}}{}
can specify low and high tags to keep float or int within a range.

\end{fulllineitems}

\index{private\_property() (in module taref.core.atom\_extension)}

\begin{fulllineitems}
\phantomsection\label{core_doc/atom_extension:taref.core.atom_extension.private_property}\pysiglinewithargsret{\code{taref.core.atom\_extension.}\bfcode{private\_property}}{\emph{fget}}{}
A decorator which converts a function into a cached Property tagged as private.
Improves performance greatly over property!

\end{fulllineitems}

\index{reset\_properties() (in module taref.core.atom\_extension)}

\begin{fulllineitems}
\phantomsection\label{core_doc/atom_extension:taref.core.atom_extension.reset_properties}\pysiglinewithargsret{\code{taref.core.atom\_extension.}\bfcode{reset\_properties}}{\emph{obj}}{}
resets all  properties that are in all\_params

\end{fulllineitems}

\index{set\_all\_tags() (in module taref.core.atom\_extension)}

\begin{fulllineitems}
\phantomsection\label{core_doc/atom_extension:taref.core.atom_extension.set_all_tags}\pysiglinewithargsret{\code{taref.core.atom\_extension.}\bfcode{set\_all\_tags}}{\emph{obj}, \emph{**kwargs}}{}
Shortcut to use Atom's tag functionality to set metadata on members not marked private, i.e. all\_params. This is an easy way to set the same tag on all params

\end{fulllineitems}

\index{set\_attr() (in module taref.core.atom\_extension)}

\begin{fulllineitems}
\phantomsection\label{core_doc/atom_extension:taref.core.atom_extension.set_attr}\pysiglinewithargsret{\code{taref.core.atom\_extension.}\bfcode{set\_attr}}{\emph{self}, \emph{name}, \emph{value}, \emph{**kwargs}}{}
utility function for setting tags while setting value

\end{fulllineitems}

\index{set\_log() (in module taref.core.atom\_extension)}

\begin{fulllineitems}
\phantomsection\label{core_doc/atom_extension:taref.core.atom_extension.set_log}\pysiglinewithargsret{\code{taref.core.atom\_extension.}\bfcode{set\_log}}{\emph{obj}, \emph{name}, \emph{value}}{}
called when parameter of given name is set to value i.e. instr.parameter=value. Customized messages for different types. Also saves data

\end{fulllineitems}

\index{set\_tag() (in module taref.core.atom\_extension)}

\begin{fulllineitems}
\phantomsection\label{core_doc/atom_extension:taref.core.atom_extension.set_tag}\pysiglinewithargsret{\code{taref.core.atom\_extension.}\bfcode{set\_tag}}{\emph{obj}, \emph{name}, \emph{**kwargs}}{}
sets the tag of a member using Atom's built in tag functionality

\end{fulllineitems}

\index{set\_value\_map() (in module taref.core.atom\_extension)}

\begin{fulllineitems}
\phantomsection\label{core_doc/atom_extension:taref.core.atom_extension.set_value_map}\pysiglinewithargsret{\code{taref.core.atom\_extension.}\bfcode{set\_value\_map}}{\emph{obj}, \emph{name}, \emph{value}}{}
checks floats and ints for low/high limits and automaps an Enum when setting. Not working for List?

\end{fulllineitems}

\index{tag\_Callable (class in taref.core.atom\_extension)}

\begin{fulllineitems}
\phantomsection\label{core_doc/atom_extension:taref.core.atom_extension.tag_Callable}\pysiglinewithargsret{\strong{class }\code{taref.core.atom\_extension.}\bfcode{tag\_Callable}}{\emph{**kwargs}}{}
disposable decorator class that returns a Callable tagged with kwargs

\end{fulllineitems}

\index{tag\_Property (class in taref.core.atom\_extension)}

\begin{fulllineitems}
\phantomsection\label{core_doc/atom_extension:taref.core.atom_extension.tag_Property}\pysiglinewithargsret{\strong{class }\code{taref.core.atom\_extension.}\bfcode{tag\_Property}}{\emph{cached=True}, \emph{**kwargs}}{}
disposable decorator class that returns a cached Property tagged with kwargs

\end{fulllineitems}



\subsection{shower}
\label{core_doc/shower:shower}\label{core_doc/shower:module-taref.core.shower}\label{core_doc/shower::doc}\index{taref.core.shower (module)}
Created on Mon Aug 24 12:38:54 2015

@author: thomasaref
\index{shower() (in module taref.core.shower)}

\begin{fulllineitems}
\phantomsection\label{core_doc/shower:taref.core.shower.shower}\pysiglinewithargsret{\code{taref.core.shower.}\bfcode{shower}}{\emph{*agents}, \emph{**kwargs}}{}
A powerful showing function for any Atom object(s) specified in agents.
Checks if an object has a view\_window and otherwise uses a default window for the object.
\begin{description}
\item[{Checks kwargs for particular keywords:}] \leavevmode\begin{itemize}
\item {} 
\code{start\_it}: boolean representing whether to go through first time setup prior to starting app

\item {} 
\code{app}: defaults to existing QtApplication instance and will default to a new instance if none exists

\end{itemize}

chief\_cls: if not included defaults to the first agent and defaults to Backbone if no agents are passed.
show\_log: shows the log\_window of chief\_cls if it has one, defaults to not showing
show\_ipy: shows the interactive\_window of chief\_cls if it has one, defaults to not showing
show\_code: shows the code\_window of chief\_cls if it has one, defaults to not showing

\end{description}

shower also provides a chief\_window (generally for controlling which agents are visible) which defaults to Backbone's chief\_window
if chief\_cls does not have one. attributes of chief\_window can be modified with the remaining kwargs

\end{fulllineitems}



\subsection{backbone}
\label{core_doc/backbone:backbone}\label{core_doc/backbone::doc}\label{core_doc/backbone:module-taref.core.backbone}\index{taref.core.backbone (module)}
Created on Tue Jul  7 21:52:51 2015

@author: thomasaref
\index{Backbone (class in taref.core.backbone)}

\begin{fulllineitems}
\phantomsection\label{core_doc/backbone:taref.core.backbone.Backbone}\pysiglinewithargsret{\strong{class }\code{taref.core.backbone.}\bfcode{Backbone}}{\emph{*args}, \emph{**kwargs}}{}
Class combining primary functions for viewer operation.
Extends \_\_init\_\_ to allow extra setup.
extends \_\_setattr\_\_ to perform low/high check on params
\index{all\_main\_params (taref.core.backbone.Backbone attribute)}

\begin{fulllineitems}
\phantomsection\label{core_doc/backbone:taref.core.backbone.Backbone.all_main_params}\pysigline{\bfcode{all\_main\_params}}
A Member which behaves similar to a Python property.

\end{fulllineitems}

\index{all\_params (taref.core.backbone.Backbone attribute)}

\begin{fulllineitems}
\phantomsection\label{core_doc/backbone:taref.core.backbone.Backbone.all_params}\pysigline{\bfcode{all\_params}}
A Member which behaves similar to a Python property.

\end{fulllineitems}

\index{app (taref.core.backbone.Backbone attribute)}

\begin{fulllineitems}
\phantomsection\label{core_doc/backbone:taref.core.backbone.Backbone.app}\pysigline{\bfcode{app}\strong{ = None}}
\end{fulllineitems}

\index{call\_func() (taref.core.backbone.Backbone method)}

\begin{fulllineitems}
\phantomsection\label{core_doc/backbone:taref.core.backbone.Backbone.call_func}\pysiglinewithargsret{\bfcode{call\_func}}{\emph{name}, \emph{**kwargs}}{}
calls a func using keyword assignments. If name corresponds to a Property, calls the get func.
otherwise, if name\_mangled func ``\_get\_''+name exists, calls that. Finally calls just the name if these are not the case

\end{fulllineitems}

\index{chief\_window (taref.core.backbone.Backbone attribute)}

\begin{fulllineitems}
\phantomsection\label{core_doc/backbone:taref.core.backbone.Backbone.chief_window}\pysigline{\bfcode{chief\_window}\strong{ = \textless{}taref.core.agent\_e.BasicView object\textgreater{}}}
\end{fulllineitems}

\index{code\_window (taref.core.backbone.Backbone attribute)}

\begin{fulllineitems}
\phantomsection\label{core_doc/backbone:taref.core.backbone.Backbone.code_window}\pysigline{\bfcode{code\_window}\strong{ = \textless{}taref.core.interactive\_e.CodeWindow object\textgreater{}}}
\end{fulllineitems}

\index{extra\_setup() (taref.core.backbone.Backbone method)}

\begin{fulllineitems}
\phantomsection\label{core_doc/backbone:taref.core.backbone.Backbone.extra_setup}\pysiglinewithargsret{\bfcode{extra\_setup}}{\emph{param}, \emph{typer}}{}
Performs extra setup during initialization where param is name of parameter and typer is it's Atom type.
Can be customized in child classes. default extra setup handles units, auto tags low and high for Ranges, and makes Callables into instancemethods

\end{fulllineitems}

\index{instancemethod() (taref.core.backbone.Backbone method)}

\begin{fulllineitems}
\phantomsection\label{core_doc/backbone:taref.core.backbone.Backbone.instancemethod}\pysiglinewithargsret{\bfcode{instancemethod}}{\emph{func}}{}
decorator for adding instancemethods defined outside of class

\end{fulllineitems}

\index{interactive\_window (taref.core.backbone.Backbone attribute)}

\begin{fulllineitems}
\phantomsection\label{core_doc/backbone:taref.core.backbone.Backbone.interactive_window}\pysigline{\bfcode{interactive\_window}\strong{ = \textless{}taref.core.interactive\_e.InteractiveWindow object\textgreater{}}}
\end{fulllineitems}

\index{log\_window (taref.core.backbone.Backbone attribute)}

\begin{fulllineitems}
\phantomsection\label{core_doc/backbone:taref.core.backbone.Backbone.log_window}\pysigline{\bfcode{log\_window}\strong{ = \textless{}taref.core.log\_e.LogWindow object\textgreater{}}}
\end{fulllineitems}

\index{main\_params (taref.core.backbone.Backbone attribute)}

\begin{fulllineitems}
\phantomsection\label{core_doc/backbone:taref.core.backbone.Backbone.main_params}\pysigline{\bfcode{main\_params}}
A Member which behaves similar to a Python property.

\end{fulllineitems}

\index{property\_dict (taref.core.backbone.Backbone attribute)}

\begin{fulllineitems}
\phantomsection\label{core_doc/backbone:taref.core.backbone.Backbone.property_dict}\pysigline{\bfcode{property\_dict}}
A Member which behaves similar to a Python property.

\end{fulllineitems}

\index{property\_names (taref.core.backbone.Backbone attribute)}

\begin{fulllineitems}
\phantomsection\label{core_doc/backbone:taref.core.backbone.Backbone.property_names}\pysigline{\bfcode{property\_names}}
A Member which behaves similar to a Python property.

\end{fulllineitems}

\index{property\_values (taref.core.backbone.Backbone attribute)}

\begin{fulllineitems}
\phantomsection\label{core_doc/backbone:taref.core.backbone.Backbone.property_values}\pysigline{\bfcode{property\_values}}
A Member which behaves similar to a Python property.

\end{fulllineitems}

\index{reserved\_names (taref.core.backbone.Backbone attribute)}

\begin{fulllineitems}
\phantomsection\label{core_doc/backbone:taref.core.backbone.Backbone.reserved_names}\pysigline{\bfcode{reserved\_names}}
A Member which behaves similar to a Python property.

\end{fulllineitems}

\index{unit\_dict (taref.core.backbone.Backbone attribute)}

\begin{fulllineitems}
\phantomsection\label{core_doc/backbone:taref.core.backbone.Backbone.unit_dict}\pysigline{\bfcode{unit\_dict}\strong{ = \{`c': 0.01, `\%': 0.01, `nm': 1e-09, `G': 1000000000.0, `mm': 0.001, `M': 1000000.0, `k': 1000.0, `m': 0.001, `kHz': 1000.0, `km': 1000.0, `n': 1e-09, `um': 1e-06, `cm': 0.01, `u': 1e-06, `GHz': 1000000000.0, `MHz': 1000000.0, `mW': 0.001\}}}
\end{fulllineitems}

\index{view\_window (taref.core.backbone.Backbone attribute)}

\begin{fulllineitems}
\phantomsection\label{core_doc/backbone:taref.core.backbone.Backbone.view_window}\pysigline{\bfcode{view\_window}}
A Member which behaves similar to a Python property.

\end{fulllineitems}


\end{fulllineitems}



\subsection{agent}
\label{core_doc/agent:module-taref.core.agent}\label{core_doc/agent::doc}\label{core_doc/agent:agent}\index{taref.core.agent (module)}
Created on Sat Jul  4 13:03:26 2015

@author: thomasaref
\index{Agent (class in taref.core.agent)}

\begin{fulllineitems}
\phantomsection\label{core_doc/agent:taref.core.agent.Agent}\pysiglinewithargsret{\strong{class }\code{taref.core.agent.}\bfcode{Agent}}{\emph{**kwargs}}{}
Agents use primarily setattr to log changes to params
\index{base\_name (taref.core.agent.Agent attribute)}

\begin{fulllineitems}
\phantomsection\label{core_doc/agent:taref.core.agent.Agent.base_name}\pysigline{\bfcode{base\_name}\strong{ = `agent'}}
\end{fulllineitems}

\index{extra\_setup() (taref.core.agent.Agent method)}

\begin{fulllineitems}
\phantomsection\label{core_doc/agent:taref.core.agent.Agent.extra_setup}\pysiglinewithargsret{\bfcode{extra\_setup}}{\emph{param}, \emph{typer}}{}
adds observer for ContainerLists to catch changes not covered by setattr.
extra\_setup goes from Spy, not Agent to not add observers

\end{fulllineitems}

\index{log\_changes() (taref.core.agent.Agent method)}

\begin{fulllineitems}
\phantomsection\label{core_doc/agent:taref.core.agent.Agent.log_changes}\pysiglinewithargsret{\bfcode{log\_changes}}{\emph{change}}{}
a simple logger for changes not of type create or update that also resets properties

\end{fulllineitems}


\end{fulllineitems}

\index{Operative (class in taref.core.agent)}

\begin{fulllineitems}
\phantomsection\label{core_doc/agent:taref.core.agent.Operative}\pysiglinewithargsret{\strong{class }\code{taref.core.agent.}\bfcode{Operative}}{\emph{**kwargs}}{}
Adds functionality for auto showing to Backbone
\index{abort (taref.core.agent.Operative attribute)}

\begin{fulllineitems}
\phantomsection\label{core_doc/agent:taref.core.agent.Operative.abort}\pysigline{\bfcode{abort}\strong{ = False}}
\end{fulllineitems}

\index{activated() (taref.core.agent.Operative class method)}

\begin{fulllineitems}
\phantomsection\label{core_doc/agent:taref.core.agent.Operative.activated}\pysiglinewithargsret{\strong{classmethod }\bfcode{activated}}{}{}
function that runs when window is activated

\end{fulllineitems}

\index{add\_func() (taref.core.agent.Operative method)}

\begin{fulllineitems}
\phantomsection\label{core_doc/agent:taref.core.agent.Operative.add_func}\pysiglinewithargsret{\bfcode{add\_func}}{\emph{*funcs}}{}
adds functions to run\_func\_dict. functions should be a classmethod, a staticmethod
or a separate function that takes no arguments

\end{fulllineitems}

\index{agent\_dict (taref.core.agent.Operative attribute)}

\begin{fulllineitems}
\phantomsection\label{core_doc/agent:taref.core.agent.Operative.agent_dict}\pysigline{\bfcode{agent\_dict}\strong{ = OrderedDict()}}
\end{fulllineitems}

\index{base\_name (taref.core.agent.Operative attribute)}

\begin{fulllineitems}
\phantomsection\label{core_doc/agent:taref.core.agent.Operative.base_name}\pysigline{\bfcode{base\_name}\strong{ = `operative'}}
\end{fulllineitems}

\index{cls\_run\_funcs (taref.core.agent.Operative attribute)}

\begin{fulllineitems}
\phantomsection\label{core_doc/agent:taref.core.agent.Operative.cls_run_funcs}\pysigline{\bfcode{cls\_run\_funcs}}
A Member which behaves similar to a Python property.

\end{fulllineitems}

\index{desc (taref.core.agent.Operative attribute)}

\begin{fulllineitems}
\phantomsection\label{core_doc/agent:taref.core.agent.Operative.desc}\pysigline{\bfcode{desc}}
A value of type \emph{unicode}.

By default, plain strings will be promoted to unicode strings. Pass
strict=True to the constructor to enable strict unicode checking.

\end{fulllineitems}

\index{name (taref.core.agent.Operative attribute)}

\begin{fulllineitems}
\phantomsection\label{core_doc/agent:taref.core.agent.Operative.name}\pysigline{\bfcode{name}}
A value of type \emph{unicode}.

By default, plain strings will be promoted to unicode strings. Pass
strict=True to the constructor to enable strict unicode checking.

\end{fulllineitems}

\index{run\_all() (taref.core.agent.Operative class method)}

\begin{fulllineitems}
\phantomsection\label{core_doc/agent:taref.core.agent.Operative.run_all}\pysiglinewithargsret{\strong{classmethod }\bfcode{run\_all}}{}{}
runs all functions added in run\_func\_dict''. Can be included in cls\_run\_funcs

\end{fulllineitems}

\index{run\_func\_dict (taref.core.agent.Operative attribute)}

\begin{fulllineitems}
\phantomsection\label{core_doc/agent:taref.core.agent.Operative.run_func_dict}\pysigline{\bfcode{run\_func\_dict}\strong{ = OrderedDict()}}
\end{fulllineitems}

\index{show() (taref.core.agent.Operative method)}

\begin{fulllineitems}
\phantomsection\label{core_doc/agent:taref.core.agent.Operative.show}\pysiglinewithargsret{\bfcode{show}}{\emph{*args}, \emph{**kwargs}}{}
\end{fulllineitems}


\end{fulllineitems}

\index{Spy (class in taref.core.agent)}

\begin{fulllineitems}
\phantomsection\label{core_doc/agent:taref.core.agent.Spy}\pysiglinewithargsret{\strong{class }\code{taref.core.agent.}\bfcode{Spy}}{\emph{**kwargs}}{}
Spies uses observers to log all changes to params
\index{base\_name (taref.core.agent.Spy attribute)}

\begin{fulllineitems}
\phantomsection\label{core_doc/agent:taref.core.agent.Spy.base_name}\pysigline{\bfcode{base\_name}\strong{ = `spy'}}
\end{fulllineitems}

\index{extra\_setup() (taref.core.agent.Spy method)}

\begin{fulllineitems}
\phantomsection\label{core_doc/agent:taref.core.agent.Spy.extra_setup}\pysiglinewithargsret{\bfcode{extra\_setup}}{\emph{param}, \emph{typer}}{}
adds log\_changes observer to all params

\end{fulllineitems}

\index{log\_changes() (taref.core.agent.Spy method)}

\begin{fulllineitems}
\phantomsection\label{core_doc/agent:taref.core.agent.Spy.log_changes}\pysiglinewithargsret{\bfcode{log\_changes}}{\emph{change}}{}
a simple logger for all changes and to reset properties

\end{fulllineitems}


\end{fulllineitems}



\subsection{extra\_setup}
\label{core_doc/extra_setup::doc}\label{core_doc/extra_setup:module-taref.core.extra_setup}\label{core_doc/extra_setup:extra-setup}\index{taref.core.extra\_setup (module)}
Created on Mon Jan 25 22:29:11 2016

@author: thomasaref
\index{extra\_setup() (in module taref.core.extra\_setup)}

\begin{fulllineitems}
\phantomsection\label{core_doc/extra_setup:taref.core.extra_setup.extra_setup}\pysiglinewithargsret{\code{taref.core.extra\_setup.}\bfcode{extra\_setup}}{\emph{self}, \emph{param}, \emph{typer}}{}
sets up property\_fs, ranges, and units

\end{fulllineitems}

\index{fset\_maker() (in module taref.core.extra\_setup)}

\begin{fulllineitems}
\phantomsection\label{core_doc/extra_setup:taref.core.extra_setup.fset_maker}\pysiglinewithargsret{\code{taref.core.extra\_setup.}\bfcode{fset\_maker}}{\emph{obj}, \emph{fget}, \emph{name}}{}
\end{fulllineitems}

\index{param\_decider() (in module taref.core.extra\_setup)}

\begin{fulllineitems}
\phantomsection\label{core_doc/extra_setup:taref.core.extra_setup.param_decider}\pysiglinewithargsret{\code{taref.core.extra\_setup.}\bfcode{param\_decider}}{\emph{obj}, \emph{value}, \emph{param}, \emph{pname}}{}
\end{fulllineitems}

\index{property\_func() (in module taref.core.extra\_setup)}

\begin{fulllineitems}
\phantomsection\label{core_doc/extra_setup:taref.core.extra_setup.property_func}\pysiglinewithargsret{\code{taref.core.extra\_setup.}\bfcode{property\_func}}{\emph{func}}{}
\end{fulllineitems}

\index{tagged\_property (class in taref.core.extra\_setup)}

\begin{fulllineitems}
\phantomsection\label{core_doc/extra_setup:taref.core.extra_setup.tagged_property}\pysiglinewithargsret{\strong{class }\code{taref.core.extra\_setup.}\bfcode{tagged\_property}}{\emph{cached=True}, \emph{**kwargs}}{}
\end{fulllineitems}



\subsection{fund\_core\_e}
\label{core_doc/fund_core_e:fund-core-e}\label{core_doc/fund_core_e::doc}

\section{physics}
\label{physics_doc/physics:physics}\label{physics_doc/physics::doc}
Contents:


\subsection{fundamentals}
\label{physics_doc/fundamentals:fundamentals}\label{physics_doc/fundamentals::doc}\label{physics_doc/fundamentals:module-taref.physics.fundamentals}\index{taref.physics.fundamentals (module)}
Created on Thu Jan 29 21:05:03 2015

@author: thomasaref

Gathers all useful constants and functions in one location. Also initiates a default log\_file.
\index{sinc() (in module taref.physics.fundamentals)}

\begin{fulllineitems}
\phantomsection\label{physics_doc/fundamentals:taref.physics.fundamentals.sinc}\pysiglinewithargsret{\code{taref.physics.fundamentals.}\bfcode{sinc}}{\emph{X}}{}
sinc function which doesn't autoinclude pi

\end{fulllineitems}

\index{sinc\_sq() (in module taref.physics.fundamentals)}

\begin{fulllineitems}
\phantomsection\label{physics_doc/fundamentals:taref.physics.fundamentals.sinc_sq}\pysiglinewithargsret{\code{taref.physics.fundamentals.}\bfcode{sinc\_sq}}{\emph{X}}{}
sinc squared which doesn't autoinclude pi

\end{fulllineitems}



\chapter{Indices and tables}
\label{index:indices-and-tables}\begin{itemize}
\item {} 
\DUspan{xref,std,std-ref}{genindex}

\item {} 
\DUspan{xref,std,std-ref}{modindex}

\item {} 
\DUspan{xref,std,std-ref}{search}

\end{itemize}


\renewcommand{\indexname}{Python Module Index}
\begin{theindex}
\def\bigletter#1{{\Large\sffamily#1}\nopagebreak\vspace{1mm}}
\bigletter{t}
\item {\texttt{taref.core.agent}}, \pageref{core_doc/agent:module-taref.core.agent}
\item {\texttt{taref.core.atom\_extension}}, \pageref{core_doc/atom_extension:module-taref.core.atom_extension}
\item {\texttt{taref.core.backbone}}, \pageref{core_doc/backbone:module-taref.core.backbone}
\item {\texttt{taref.core.extra\_setup}}, \pageref{core_doc/extra_setup:module-taref.core.extra_setup}
\item {\texttt{taref.core.shower}}, \pageref{core_doc/shower:module-taref.core.shower}
\item {\texttt{taref.physics.fundamentals}}, \pageref{physics_doc/fundamentals:module-taref.physics.fundamentals}
\end{theindex}

\renewcommand{\indexname}{Index}
\printindex
\end{document}
